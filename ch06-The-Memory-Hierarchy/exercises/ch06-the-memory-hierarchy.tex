%%%%%%%%%%%%%%%%%%%%%%%%%%%%%%%%%%%%%%%%%%%%%%%%%%%%%%%%%%%%%%%
% Welcome to the MAT320 Homework template on Overleaf -- just edit your
% LaTeX on the left, and we'll compile it for you on the right.
%%%%%%%%%%%%%%%%%%%%%%%%%%%%%%%%%%%%%%%%%%%%%%%%%%%%%%%%%%%%%%%
% --------------------------------------------------------------
% Based on a homework template by Dana Ernst.
% --------------------------------------------------------------
% This is all preamble stuff that you don't have to worry about.
% Head down to where it says "Start here"
% --------------------------------------------------------------

\documentclass[12pt]{article}

\usepackage{graphicx}
\graphicspath{{./images/}}
\usepackage{textcomp} % cent symbol, such as \textcent
\usepackage[margin=1in]{geometry} 
\usepackage{amsmath,amsthm,amssymb}
\usepackage{cancel}
\usepackage{mathtools} % ceiling function
\DeclarePairedDelimiter{\ceil}{\lceil}{\rceil}
% https://tex.stackexchange.com/questions/146306/how-to-make-horizontal-lists
\usepackage[inline]{enumitem} % allows using letters in enumerate list environment

% source: https://stackoverflow.com/questions/3175105/inserting-code-in-this-latex-document-with-indentation

\usepackage{listings}
\usepackage{color}

\definecolor{dkgreen}{rgb}{0,0.6,0}
\definecolor{gray}{rgb}{0.5,0.5,0.5}
\definecolor{mauve}{rgb}{0.58,0,0.82}

\lstset{frame=tb,
	language=C, % language for code listing
	aboveskip=3mm,
	belowskip=3mm,
	showstringspaces=false,
	columns=flexible,
	basicstyle={\small\ttfamily},
	numbers=none,
	numberstyle=\tiny\color{gray},
	keywordstyle=\color{blue},
	commentstyle=\color{dkgreen},
	stringstyle=\color{mauve},
	breaklines=true,
	breakatwhitespace=true,
	tabsize=4
}

\newcommand{\N}{\mathbb{N}}
\newcommand{\Z}{\mathbb{Z}}

\newenvironment{ex}[2][Exercise]{\begin{trivlist}
		\item[\hskip \labelsep {\bfseries #1}\hskip \labelsep {\bfseries #2.}]}{\end{trivlist}}

\newenvironment{sol}[1][Solution]{\begin{trivlist}
		\item[\hskip \labelsep {\bfseries #1:}]}{\end{trivlist}}


\begin{document}

% --------------------------------------------------------------
%                         Start here
% --------------------------------------------------------------

\noindent Sergio Garcia Tapia \hfill

\noindent{\small Computer Systems: A Programmer's Perspective, by Bryant and O'Hallaron} \hfill

\noindent{\small Chapter 6: The Memory Hierarchy}
\noindent\today

\subsection*{Practice Problems}

\begin{ex}{6.1}
	In the following, let $r$ be the number of rows in a DRAM array, $c$ the number of columns,
	$b_r$ the number of bits needed to address the rows, and $b_c$ the number of bits needed to
	address the columns. For each of the following DRAMs, determine the power-of-2 array dimensions
	that minimize $\max(b_r, b_c)$, the maximum number of bits needed to address the rows or
	columns of the array.
	\begin{center}
		\begin{tabular}{cccccc}
			Organization & $r$ & $c$ & $b_r$ & $b_c$ & $\max(b_r, b_c)$\\
			\hline
			$16\times 1$ & \makebox[1cm]{\hrulefill} & \makebox[1cm]{\hrulefill} & \makebox[1cm]{\hrulefill} & \makebox[1cm]{\hrulefill} & \makebox[1cm]{\hrulefill}\\
			$16\times 4$ & \makebox[1cm]{\hrulefill} & \makebox[1cm]{\hrulefill} & \makebox[1cm]{\hrulefill} & \makebox[1cm]{\hrulefill} & \makebox[1cm]{\hrulefill}\\
			$128\times 8$ & \makebox[1cm]{\hrulefill} & \makebox[1cm]{\hrulefill} & \makebox[1cm]{\hrulefill} & \makebox[1cm]{\hrulefill} & \makebox[1cm]{\hrulefill}\\
			$512\times 4$ & \makebox[1cm]{\hrulefill} & \makebox[1cm]{\hrulefill} & \makebox[1cm]{\hrulefill} & \makebox[1cm]{\hrulefill} & \makebox[1cm]{\hrulefill}\\
			$1024\times 4$ & \makebox[1cm]{\hrulefill} & \makebox[1cm]{\hrulefill} & \makebox[1cm]{\hrulefill} & \makebox[1cm]{\hrulefill} & \makebox[1cm]{\hrulefill}\\
		\end{tabular}
	\end{center}
\end{ex}

\begin{sol}
	\
	The notation $16\times 1$ means $d=16$ supercells, each of width $w=1$ bit. The number
	of rows $r$ and columns $c$ must satisfy $rc=d=16$. Since $16$ is a perfect square,
	the minimum is given when we pick the square root, so $r=c=4$. Meanwhile, $d=128$
	is not a perfect square, and we must pick either $r=8$ and $c=16$, or $r=16$ and $c=8$.
	For $d=512$, we can pick $r=32$ and $d=16$, and for $d=1024$, we can pick $r=d=32$.
	\begin{center}
		\begin{tabular}{cccccc}
			Organization & $r$ & $c$ & $b_r$ & $b_c$ & $\max(b_r, b_c)$\\
			\hline
			$16\times 1$ & $4$ & $4$ & $2$ & $2$ & $2$ \\
			$16\times 4$ & $4$ & $4$ & $2$ & $2$ & $2$ \\
			$128\times 8$ & $16$ & $8$ & $4$ & $3$ & $4$\\
			$512\times 4$ & $32$ & $16$ & $5$ & $4$ & $5$\\
			$1024\times 4$ & $32$ & $32$ & $5$ & $5$ & $5$\\
		\end{tabular}
	\end{center}
\end{sol}

\begin{ex}{6.2}
	What is the capacity of a disk with 2 platters, $10,000$ cylinders, an average of
	400 sectors per track, and 512 bytes per sector?
\end{ex}

\begin{sol}
	\
	According Section 6.12, page 592, the capacity of a rotating disk is given by
	\[
	\text{Capacity} = \frac{\#\text{ bytes}}{\text{sector}}
	\times \frac{\text{average \# sectors}}{\text{track}}
	\times \frac{\text{average \# tracks}}{\text{surface}}
	\times \frac{\#\text{ surfaces}}{\text{platter}}
	\times \frac{\#\text{ platters}}{\text{disk}}
	\]
	Recall that a cylinder is the collection of tracks on all the surfaces that are equidistant
	from the center of the spindle. Since the disk has 10,000 cylinders, this means there's an
	average of 10,000 tracks per surface. Therefore the capacity is:
	\begin{align}
		\text{Capacity}&= \frac{512 \text{ bytes}}{\text{sector}}
		\times \frac{400 \text{ sectors}}{\text{track}}
		\times \frac{10,000\text{ tracks}}{\text{surface}}
		\times \frac{2 \text{ surfaces}}{\text{platter}}
		\times \frac{2 \text{ platters}}{\text{disk}}\\
		&=8.192\times 10^{9}\text{ bytes}\\
		&=8.192\text{ GB}
	\end{align}
\end{sol}

\begin{ex}{6.3}
	Estimate the average time (in ms) to access a sector on the following disk:
	\begin{center}
		\begin{tabular}{ll}
			Parameter & Value\\
			\hline
			Rotational Rate & 15,000 RPM\\
			$T_{\text{avg seek}}$ & 8 ms\\
			Average number of sectors/track & 500
		\end{tabular}
	\end{center}
\end{ex}

\begin{sol}
	\
	The average time to access a sector on a disk is the sum of the seek time, the rotational
	latency, and the transfer time. The seek time is how long it takes the actuator arm on
	a hard disk to move to the track that contains the target sector. The rotational latency
	is how long the drive waits for the first bit of the sector to pass under the head of
	the actuator arm. The transfer time is how long it takes to read the contents of the sector.
	
	\
	The average rotational latency is given by half of the maximum rotational latency,
	which in turn is given by:
	\[
	T_{\text{max rotation}}=\frac{1}{\text{RPM}}\times \frac{60 \text{ secs}}{\text{min}}
	\]
	Therefore,
	\[
	T_{\text{avg rotation}}=\frac{1}{2}\times \frac{1 \text{ minute}}{15000 \text{ rotations}}
	\times \frac{60 \text{ secs}}{\text{min}}
	\times \frac{1000 \text{ ms}}{\text{second}}
	=2\text{ ms}
	\]
	Meanwhile, the average transfer time is given by
	\begin{align*}
	T_{\text{avg transfer}}
	&=\frac{1}{\text{RPM}}\times \frac{1}{\text{average \# sectors/track}}
	\times \frac{60\text{ secs}}{\text{min}}\\
	&=\frac{\text{minute}}{15,000 \text{ rotations}}
	\times \frac{1}{500 \text{ sectors/track}}
	\times \frac{60 \text{ secs}}{\text{min}}
	\times \frac{1000 \text{ ms}}{\text{second}}\\
	&=0.008\text{ ms}
	\end{align*}

	Therefore we have, the average access time is about
	$T_{\text{access}}=8\text{ ms}+2\text{ ms}+0.008\text{ ms}=10.008\text{ ms}$.
\end{sol}

\begin{ex}{6.4}
	Suppose that a 1 MB file consisting of 512-byte logical blocks is stored on a disk drive with
	the following characteristics:
	\begin{center}
		\begin{tabular}{ll}
			Parameter & Value\\
			\hline
			Rotational rate & 10,000 RPM\\
			$T_{\text{avg seek}}$ & 5 ms\\
			Average number of sectors/track & 1,000\\
			Surfaces & 4\\
			Sector size & 512 bytes
		\end{tabular}
	\end{center}
	For each case below, suppose that a program reads the logical blocks of the file sequentially,
	one after the other, and that the time to position the head over the first block is
	$T_{\text{avg seek}} + T_{\text{avg rotation}}$.
	\begin{enumerate}[label=(\alph*)]
		\item \emph{Best case}: Estimate the optimal time (in ms) required to read the file given
		the best possible mapping of logical blocks to disk sectors (i.e., sequential).
		\item \emph{Random case}: Estimate the time (in ms) required to read the file if blocks
		are mapped randomly to disk sectors.
	\end{enumerate}
\end{ex}

\begin{sol}
	\
	\begin{enumerate}[label=(\alph*)]
		\item
		\item 1 MB is equivalent to 1000 KB, so with a logical block size of 512 bytes, it will
		take up about 2000 logical blocks on disk. Since the size of a block equals the size of a
		sector, and there are 1000 tracks per sector on average for the given disk, this means
		that a sector can hold about 1000 blocks. In the best case, the data for the file is
		mapped onto 2 tracks on two surfaces on the same platter with the increasing order of the logical blocks matching that of the sector. After the initial delay of
		$T_{\text{avg seek}} + T_{\text{avg rotation}}$ to place the head on the first block, it
		will take about $T_{\text{avg transfer}}$ to transfer all bytes in the sector, and about
		$\frac{1}{1000}T_{\text{max rotation}}$ to get to the next sector (logical block) on the
		same track to continue. These latter two steps are performed 1000 times per track. Since
		the two surfaces are on the same platter, the head is already in place to start
		transferring the data from the second surface. Therefore, the optimal time would be about:
		\begin{align*}
		T_{\text{access, optimal}}
		&=T_{\text{avg seek}} + T_{\text{avg rotation}}
		+2\cdot 1000\cdot T_{\text{avg transfer}}\\
		&=T_{\text{avg seek}} + T_{\text{max rotation}} + 2000\cdot T_{\text{avg transfer}}
		\end{align*}
		
		The maximum rotational latency for the given disk is about
		\begin{align*}
			T_{\text{maximum rotation}}
			=
			\times \frac{1 \text{ minute}}{10000 \text{ rotations}}
			\times \frac{60 \text{ secs}}{\text{min}}
			\times \frac{1000 \text{ ms}}{\text{second}}
			=6\text{ ms}
		\end{align*}
		The average transfer latency is about
		\begin{align*}
			T_{\text{avg transfer}}
			&=\frac{\text{minute}}{10,000 \text{ rotations}}
			\times \frac{1}{1,000 \text{ sectors/track}}
			\times \frac{60 \text{ secs}}{\text{min}}
			\times \frac{1000 \text{ ms}}{\text{second}}\\
			&=0.006\text{ ms}
		\end{align*}
		Altogether, the optimal access time is about
		\begin{align*}
			T_{\text{access, optimal}}
			&=5\text{ ms} + \frac{1}{2} \cdot 6 \text{ ms} + 2000\cdot 0.006\text{ ms}\\
			&=20\text{ ms}
		\end{align*}
		\item For a random case, we may not have contiguous blocks for the file. The 2000
		blocks for the file may be split evenly among all 4 surfaces, meaning there is about
		500 blocks per surface. If each block is on a different track, it may take about
		\begin{align*}
			T_{\text{access, avg}}&=
			2000\times \left( T_{\text{avg seek}} + T_{\text{avg rotation}} + T_{\text{avg
					transfer}} + T_{\text{access}}\right)\\
			&=2000\times (5 + \text{ ms} + 3 \text{ ms} + 0.006\text{ ms})\\
			&=16012\text{ ms}\\
			&\approx 16 \text{ seconds}
		\end{align*}
	\end{enumerate}
\end{sol}

\begin{ex}{6.5}
	As we have seen, a potential drawback of SSDs is that the underlying flash memory
	can wear out. For example, for the SSD in Figure 6.14 (shown below), Intel guarantees
	about 12 petabytes ($128\times 10^{15}$ bytes) of writes before the drive wears out:
	\begin{center}
		\begin{tabular}{ll|ll}
			\hline
			\multicolumn{2}{c}{Reads} &  \multicolumn{2}{c}{Writes}\\
			\hline
			Sequential read throughput & 550 MB/s
			& Sequential write throughput & 470 MB/s\\
			
			Random read throughput (IOPS) & 89,000 IOPS
			& Random write throughput (IOPS) & 74,000 IOPS\\
			
			Random read throughput (MB/s) & 365 MB/s
			& Random write throughput (MB/s) & 303 MB/s\\
			
			Avg. sequential read access time & 50 $\mu$s
			& Avg. sequential write access time & 60 $\mu$s\\
			\hline
		\end{tabular}
	\end{center}
	
	Given this assumption, estimate the lifetime (in years) of this SSD for the following	
	workloads:
	\begin{enumerate}[label=(\alph*)]
		\item \emph{Worst case for sequential writes}: The SSD is written to continuously
		at a rate of 470 MB/s (the average sequential write throughput of the device).
		\item \emph{Worst case for random writes}: The SSD is written continuously at a
		rate of 303 MB/s (the average random write throughput of the device).
		\item \emph{Average case}: The SSD is written to at a rate of 20 GB/day (the
		average daily write rate assumed by some computer manufacturers in their mobile
		computer workload simulations).
	\end{enumerate}
\end{ex}

\begin{sol}
	\
	\begin{enumerate}[label=(\alph*)]
		\item It would take
		\begin{align*}
			T_{\text{worst seq. writes}}&=
			128\times 10^{15}\text{ bytes}
			\cdot \frac{1 \text{ MB}}{10^6 \text{ bytes}}
			\cdot \frac{1 \text{ second}}{470 \text{ MB}}
			\cdot \frac{1 \text{ day}}{86400 \text{ seconds}}
			\cdot \frac{1 \text{ year}}{365 \text{ days}}\\
			&\approx 8.63 \text{ years}
		\end{align*}
		\item 
		\begin{align*}
			T_{\text{worst ran. writes}}&=
			128\times 10^{15}\text{ bytes}
			\cdot \frac{1 \text{ MB}}{10^6 \text{ bytes}}
			\cdot \frac{1 \text{ second}}{303 \text{ MB}}
			\cdot \frac{1 \text{ day}}{86400 \text{ seconds}}
			\cdot \frac{1 \text{ year}}{365 \text{ days}}\\
			&\approx 13.4 \text{ years}
		\end{align*}
		\item 
		\begin{align*}
			T_{\text{avg}}&=
			128\times 10^{15}\text{ bytes}
			\cdot \frac{1 \text{ GB}}{10^9 \text{ bytes}}
			\cdot \frac{1 \text{ day}}{20 \text{ GB}}
			\cdot \frac{1 \text{ year}}{365 \text{ days}}\\
			&\approx 17534 \text{ years}
		\end{align*}
	\end{enumerate}
\end{sol}

\begin{ex}{6.6}
	Using the data from the years 2005 to 2015 in Figure 6.15(c) on page 603, estimate the
	year when you will be able to buy a petabyte ($10^{15}$ bytes) of rotating disk storage
	for $\$500$ bytes. Assume actual dollars (no inflation).
\end{ex}

\begin{sol}
	\
	Expressed in dollars per GB, the cost would be
	\begin{align*}
		&\frac{\$500}{10^{15}\text{ bytes}}
		\cdot \frac{10^{9} \text{ byte}}{1 \text{ GB}}
		=\frac{\$0.0005}{\text{GB}}
	\end{align*}
	According to the table, the price for rotating disk storage in dollars per GB
	for the years 2005, 2010, and 2015 were \$5/GB, \$0.3/GB, and \$0.03/GB,
	indicating a decrease by a factor between 10 and 16.67. It would therefore
	take at least 5 years but no more than 10 years, so around 2025.
\end{sol}

\end{document}
