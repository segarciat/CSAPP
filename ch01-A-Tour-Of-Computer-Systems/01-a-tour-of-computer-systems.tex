%%%%%%%%%%%%%%%%%%%%%%%%%%%%%%%%%%%%%%%%%%%%%%%%%%%%%%%%%%%%%%%
% Welcome to the MAT320 Homework template on Overleaf -- just edit your
% LaTeX on the left, and we'll compile it for you on the right.
%%%%%%%%%%%%%%%%%%%%%%%%%%%%%%%%%%%%%%%%%%%%%%%%%%%%%%%%%%%%%%%
% --------------------------------------------------------------
% Based on a homework template by Dana Ernst.
% --------------------------------------------------------------
% This is all preamble stuff that you don't have to worry about.
% Head down to where it says "Start here"
% --------------------------------------------------------------

\documentclass[12pt]{article}

\usepackage{graphicx}
\graphicspath{{./images/}}
\usepackage{textcomp} % cent symbol, such as \textcent
\usepackage[margin=1in]{geometry} 
\usepackage{amsmath,amsthm,amssymb}
\usepackage{mathtools} % ceiling function
\DeclarePairedDelimiter{\ceil}{\lceil}{\rceil}
% https://tex.stackexchange.com/questions/146306/how-to-make-horizontal-lists
\usepackage[inline]{enumitem} % allows using letters in enumerate list environment

% source: https://stackoverflow.com/questions/3175105/inserting-code-in-this-latex-document-with-indentation

\usepackage{listings}
\usepackage{color}

\definecolor{dkgreen}{rgb}{0,0.6,0}
\definecolor{gray}{rgb}{0.5,0.5,0.5}
\definecolor{mauve}{rgb}{0.58,0,0.82}

\lstset{frame=tb,
	language=VHDL, % language for code listing
	aboveskip=3mm,
	belowskip=3mm,
	showstringspaces=false,
	columns=flexible,
	basicstyle={\small\ttfamily},
	numbers=none,
	numberstyle=\tiny\color{gray},
	keywordstyle=\color{blue},
	commentstyle=\color{dkgreen},
	stringstyle=\color{mauve},
	breaklines=true,
	breakatwhitespace=true,
	tabsize=4
}

\newcommand{\N}{\mathbb{N}}
\newcommand{\Z}{\mathbb{Z}}

\newenvironment{ex}[2][Exercise]{\begin{trivlist}
		\item[\hskip \labelsep {\bfseries #1}\hskip \labelsep {\bfseries #2.}]}{\end{trivlist}}

\newenvironment{sol}[1][Solution]{\begin{trivlist}
		\item[\hskip \labelsep {\bfseries #1:}]}{\end{trivlist}}


\begin{document}

% --------------------------------------------------------------
%                         Start here
% --------------------------------------------------------------

\noindent Sergio Garcia Tapia \hfill

\noindent{\small Computer Systems: A Programmer's Perspective, by Bryant and O'Hallaron} \hfill

\noindent{\small Chapter 1: A Tour of Computer Systems} \hfill 

\noindent\today

\subsection*{Practice Problems}
\begin{ex}{1.1}
	Suppose  you  work as a truck driver, and you have been hired to carry
	a load of potatoes from Boise, Idaho, to Minneaopolis, Minnesota, a
	total distance of 2,500 kilometers. You estimate you can average
	100 km/hr driving within the speed limits, requiring a total of 25
	hours for the trip.
	\begin{enumerate}[label=(\alph*)]
		\item You hear on the news that Montana has just abolished its
		speed limit, which constitutes 1,500 km of the trip. Your
		truck can travel at 150 km/hr. What will be your speedup
		for the trip?
		\item You can buy a new turbocharger for your truck at
		www.fasttrucks.com. They stock a variety of models, but
		the fast you go, the more it will cost. How fast must you travel
		through Montana to get an overall speedup of your trip of $1.67\times$?
	\end{enumerate}
\end{ex}

\begin{sol}
	\
	Recall that Amdhal's Law says if part of a system takes a fraction
	$\alpha$ of the overall time, and if we speed up by a factor of $k$,
	then the overall system speedup is given by
	\begin{align*}
		S=\frac{1}{(1-\alpha)+\alpha/k},
	\end{align*}
	where $\alpha$ is the fraction of a system
	\begin{enumerate}[label=(\alph*)]
		\item The distance through Montana corresponds to a fraction
		$\alpha=\frac{1500}{2500}=\frac{3}{5}$ of the overall trip.
		During the Montana part, we travel at 150 km/hr, whereas we
		travel at 100 km/hr for the rest of the trip, meaning that
		our performance improve factor through Montana is
		$k=\frac{150}{100}=\frac{3}{2}$. Therefore, the overall
		speedup is
		\begin{align*}
			S&=\frac{1}{(1-\frac{3}{5})+\frac{3}{5}\cdot \frac{2}{3}}\\
			&=\frac{1}{\frac{2}{5}+\frac{2}{5}}\\
			&=\frac{5}{4}
		\end{align*}
		That is, the speedup for the trip is $1.25\times$.
		\item To find the necessary speed of our track to obtain and
		overall speedup of $S=1.67\times$, we use Amdhal's Law to find $k$:
		\begin{align*}
			S(1-\alpha)+\frac{\alpha S}{k}&=1\\
			\frac{\alpha S}{k}&=1+S(\alpha-1)\\
			k&=\frac{\alpha S}{1+S(\alpha -1)}
		\end{align*}
		Since $1.67\approx \frac{5}{3}$, we get
		\begin{align*}
			k&=\frac{\frac{3}{5}\cdot \frac{5}{3}}{1+\frac{5}{3}\left(\frac{3}{5}-1\right)}\\
			&=\frac{1}{1-\frac{2}{3}}\\
			&=3
		\end{align*}
		Hence, we would need to travel $100\text{ km/hr } \times 3=300\text{ km/hr}$.
	\end{enumerate}
\end{sol}

\begin{ex}{2}
	The marketing department at your company has promised your customers that
	the next software release will show a $2\times$ performance improvement.
	You have been assigned the task of delivering on that promise. You have
	determined that only $80\%$ of the system can be improved. How much (i.e,
	what value of $k$) would you need to improve this part to meet the overall
	performance target?
\end{ex}

\begin{sol}
	After re-arranging Amdhal's Law in the previous question, the equation
	for $k$ became
	\begin{align*}
		k&=\frac{\alpha S}{1+S(\alpha -1)}
	\end{align*}
	Given $S=2$ and $\alpha=0.8=\frac{4}{5}$, we find the required
	performance improvement $k$:
	\begin{align*}
		k&=\frac{\frac{4}{5}\cdot 2}{1 + 2\left(\frac{4}{5}-1\right)}\\
		&=\frac{\frac{8}{5}}{\frac{3}{5}}\\
		&=\frac{8}{3}
	\end{align*}
	Hence, we need a performance improvement of about $2.67\times$.
\end{sol}


\end{document}