%%%%%%%%%%%%%%%%%%%%%%%%%%%%%%%%%%%%%%%%%%%%%%%%%%%%%%%%%%%%%%%
% Welcome to the MAT320 Homework template on Overleaf -- just edit your
% LaTeX on the left, and we'll compile it for you on the right.
%%%%%%%%%%%%%%%%%%%%%%%%%%%%%%%%%%%%%%%%%%%%%%%%%%%%%%%%%%%%%%%
% --------------------------------------------------------------
% Based on a homework template by Dana Ernst.
% --------------------------------------------------------------
% This is all preamble stuff that you don't have to worry about.
% Head down to where it says "Start here"
% --------------------------------------------------------------

\documentclass[12pt]{article}

\usepackage{graphicx}
\graphicspath{{./images/}}
\usepackage{textcomp} % cent symbol, such as \textcent
\usepackage[margin=1in]{geometry} 
\usepackage{amsmath,amsthm,amssymb}
\usepackage{cancel}
\usepackage{mathtools} % ceiling function
\DeclarePairedDelimiter{\ceil}{\lceil}{\rceil}
% https://tex.stackexchange.com/questions/146306/how-to-make-horizontal-lists
\usepackage[inline]{enumitem} % allows using letters in enumerate list environment

% source: https://stackoverflow.com/questions/3175105/inserting-code-in-this-latex-document-with-indentation

\usepackage{listings}
\usepackage{color}

\definecolor{dkgreen}{rgb}{0,0.6,0}
\definecolor{gray}{rgb}{0.5,0.5,0.5}
\definecolor{mauve}{rgb}{0.58,0,0.82}

\lstset{frame=tb,
	language=C, % language for code listing
	aboveskip=3mm,
	belowskip=3mm,
	showstringspaces=false,
	columns=flexible,
	basicstyle={\small\ttfamily},
	numbers=none,
	numberstyle=\tiny\color{gray},
	keywordstyle=\color{blue},
	commentstyle=\color{dkgreen},
	stringstyle=\color{mauve},
	breaklines=true,
	breakatwhitespace=true,
	tabsize=4
}

\newcommand{\N}{\mathbb{N}}
\newcommand{\Z}{\mathbb{Z}}

\newenvironment{ex}[2][Exercise]{\begin{trivlist}
		\item[\hskip \labelsep {\bfseries #1}\hskip \labelsep {\bfseries #2.}]}{\end{trivlist}}

\newenvironment{sol}[1][Solution]{\begin{trivlist}
		\item[\hskip \labelsep {\bfseries #1:}]}{\end{trivlist}}


\begin{document}

% --------------------------------------------------------------
%                         Start here
% --------------------------------------------------------------

\noindent Sergio Garcia Tapia \hfill

\noindent{\small Computer Systems: A Programmer's Perspective, by Bryant and O'Hallaron} \hfill

\noindent{\small Chapter 2: Representing and Manipulating Information} \hfill 

\noindent\today

\subsection*{Practice Problems}
\begin{ex}{2.1}
	Perform the following conversions:
	\begin{enumerate}[label=(\alph*)]
		\item \texttt{0x39A7F8} to binary.
		\item binary $1100100101111011$ to hexadecimal.
		\item \texttt{0xD5E4C} to binary.
		\item binary $1001101110011110110101$ to hexadecimal.
	\end{enumerate}
\end{ex}

\begin{sol}
	\
	\begin{enumerate}[label=(\alph*)]
		\item Each hexadecimal digit corresponds to a 4-bit binary number:
		\begin{center}
			\begin{tabular}{c|cccccc}
				Hexadecimal & \texttt{3} & \texttt{9} & \texttt{A} & \texttt{7} & \texttt{F} & \texttt{8}\\
				\hline
				Binary & 0011 & 1001 & 1010 & 0111 & 1111 & 1000
			\end{tabular}
		\end{center}
		When concatenated, we find that $\texttt{0x39A7F8}=0011 1001 1010 0111 1111 1000_2$.
		\item We group the number into 4-bit groups:
		\begin{center}
			\begin{tabular}{c|cccc}
				Binary      & 1100 & 1001 & 0111 & 1011\\
				\hline
				Hexadecimal & \texttt{C} &  \texttt{9} &  \texttt{7} &  \texttt{B}
			\end{tabular}
		\end{center}
		Hence, $1100 1001 0111 1011_2=\texttt{0xC97B}$.
		\item We tabulate the values:
		\begin{center}
			\begin{tabular}{c|ccccc}
				Hexadecimal & \texttt{D} & \texttt{5} & \texttt{E} & \texttt{4} & \texttt{C} \\
				\hline
				Binary & 1101 & 0101 & 1110 & 0100 & 1100
			\end{tabular}
		\end{center}
		Hence, $\texttt{0xD5E4C}=1101 0101 1110 0100 1100_2$.
		\item The table is below:
		\begin{center}
			\begin{tabular}{c|cccccc}
				10 0110 1110 0111 1011 0101
				Binary      & 10 & 0110 & 1110 & 0111 & 1011 & 0101\\
				\hline
				Hexadecimal & \texttt{2} & \texttt{6} & \texttt{E} & \texttt{7} & \texttt{B} & \texttt{5}
			\end{tabular}
		\end{center}
		Hence, $10 0110 1110 0111 1011 0101_2=\texttt{0x26E7B5}$.
	\end{enumerate}
\end{sol}

\begin{ex}{2.2}
	Fill in the blank entries in the following table, giving the decimal and
	hexadecimal representations of different powers of $2$:
	\begin{center}
		\begin{tabular}{ccc}
			$n$ & $2^n$ (decimal) & $2^n$ (hexadecimal)\\
			\hline
			9 & 512 & \texttt{0x200}\\
			19 & \makebox[1cm]{\hrulefill} & \makebox[1cm]{\hrulefill}\\
			\makebox[1cm]{\hrulefill} & 16,384 & \makebox[1cm]{\hrulefill}\\
			\makebox[1cm]{\hrulefill} & \makebox[1cm]{\hrulefill} & \texttt{0x1000}\\
			17 & \makebox[1cm]{\hrulefill} & \makebox[1cm]{\hrulefill}\\
			\makebox[1cm]{\hrulefill} & 32 & \makebox[1cm]{\hrulefill}\\
			\makebox[1cm]{\hrulefill} & \makebox[1cm]{\hrulefill} & \texttt{0x80}
		\end{tabular}
	\end{center}
\end{ex}

\begin{sol}
	As per the text, we write $n=i+4j$, where $0\leq i\leq 3$. The
	$i$ determines the leading hex bit to be $2^i$ (that is, 1, 2, 4, or 8).
	The $j$ determines the number of hexadecimal 0s thereafter. Some
	of the $n$ values from the table are below:
	\begin{align*}
		19&=3+4\cdot 4\\
		14&=2+4\cdot 3\\
		12&=0+4\cdot 3\\
		17&=1+4\cdot 4\\
		5 &=1+4\cdot 1\\
		7 &=3+4\cdot 1.
	\end{align*}
	The filled-in table follows:
	\begin{center}
		\begin{tabular}{ccc}
			$n$ & $2^n$ (decimal) & $2^n$ (hexadecimal)\\
			\hline
			9 & 512 & \texttt{0x200}\\
			19 & \textbf{524,288} & \textbf{\texttt{0x80000}}\\
			14 & 16,384 & \texttt{0x4000}\\
			12 & 4096 & \texttt{0x1000}\\
			17 & 131,072 & \texttt{0x20000}\\
			5 & 32 & \texttt{0x20}\\
			7 & 128 & \texttt{0x80}
		\end{tabular}
	\end{center}
\end{sol}

\begin{ex}{2.3}
	A single byte can be represented by 2 hexadecimal digits. Fill in the missing
	entries in the following table, giving the decimal, binary, and hexadecimal
	values of different byte patterns.
	\begin{center}
		\begin{tabular}{ccc}
			Decimal & Binary & Hexadecimal\\
			\hline
			0 & 0000 000 & \texttt{0x00}\\
			167 & \makebox[1cm]{\hrulefill} & \makebox[1cm]{\hrulefill}\\
			62 & \makebox[1cm]{\hrulefill} & \makebox[1cm]{\hrulefill}\\
			188 & \makebox[1cm]{\hrulefill} & \makebox[1cm]{\hrulefill}\\
			\makebox[1cm]{\hrulefill} & 0011 0111 & \makebox[1cm]{\hrulefill}\\
			\makebox[1cm]{\hrulefill} & 1000 1000 & \makebox[1cm]{\hrulefill}\\
			\makebox[1cm]{\hrulefill} & 1111 0011 & \makebox[1cm]{\hrulefill}\\
			\makebox[1cm]{\hrulefill} & \makebox[1cm]{\hrulefill} & \texttt{0x52}\\
			\makebox[1cm]{\hrulefill} & \makebox[1cm]{\hrulefill} & \texttt{0xAC}\\
			\makebox[1cm]{\hrulefill} & \makebox[1cm]{\hrulefill} & \texttt{0xE7}		\end{tabular}
	\end{center}
\end{ex}

\begin{sol}
	We proceed by repeatedly performing the division algorithm, taking each remainder:
	\begin{align*}
		167 &= 16\cdot 10 + 7\\
		10 &= 16 \cdot 0 +10
	\end{align*}
	Since $10_{16}=\texttt{0xA}$, we have $167_{10}=\texttt{0xA7}$. We proceed
	the same way:
	\begin{align*}
		62 &= 16 \cdot 3 + 14\\
		3 &= 16 \cdot 0 + 3
	\end{align*}
	So $62_{10}=\texttt{0x3E}$.
	\begin{align*}
		188&=16\cdot 11 + 12\\
		12 &=16\cdot 0 + 12
	\end{align*}
	So $188_{16}=\texttt{0xCC}$. By representing each hexadecimal digit with
	4 bits and concatenating them, we get the binary representation. To convert
	from hexadecimal to decimal, we multiply by the appropriate power of 16:
	\begin{align*}
		\texttt{0x37}&=3\cdot 16^1+7\cdot 16^0=55_{10}\\
		\texttt{0x88}&=7\cdot 16^1+8\cdot 16^0=136_{10}\\
		\texttt{0xF3}&=15\cdot16^1+3\cdot 16^0=243_{10}\\
		\texttt{0x52}&=5\cdot 16^1+2\cdot 16^0=82_{10}\\
		\texttt{0xAC}&=10\cdot16^1+12\cdot 16^0=172_{10}\\
		\texttt{0xE7}&=14\cdot16^1+7\cdot 16^0=231_{10}
	\end{align*}
	
	The complete table is below:
	\begin{center}
		\begin{tabular}{ccc}
			Decimal & Binary & Hexadecimal\\
			\hline
			0 & 0000 000 & \texttt{0x00}\\
			167 & 1010 0111& \texttt{0xA7}\\
			62 & 0011 1110 & \texttt{0x3E}\\
			188 & 1100 1100 & \texttt{0xCC}\\
			55 & 0011 0111 & \texttt{0x37}\\
			136 & 1000 1000 & \texttt{0x88}\\
			243 & 1111 0011 & \texttt{0xF3}\\
			82 & 0101 0010 & \texttt{0x52}\\
			172 & 1010 1100 & \texttt{0xAC}\\
			231 & 1110 0111 & \texttt{0xE7}	
		\end{tabular}
	\end{center}
\end{sol}


\begin{ex}{2.4}
	Without converting the numbers to decimal or binary, try to solve the following
	arithmetic problems, giving the answers in hexadecimal. \emph{Hint}: Just
	modify the methods you use for performing decimal addition and subtraction
	to use base 16.
	\begin{enumerate}[label=(\alph*)]
		\item \texttt{0x503c} + \texttt{0x8} = \makebox[1cm]{\hrulefill}
		\item \texttt{0x503c} - \texttt{0x40} = \makebox[1cm]{\hrulefill}
		\item \texttt{0x503c} + 64 = \makebox[1cm]{\hrulefill} 
		\item \texttt{0x50ea} - \texttt{0x503c} = \makebox[1cm]{\hrulefill} 
	\end{enumerate}
\end{ex}

\begin{sol}
	\
	\begin{enumerate}[label=(\alph*)]
		\item 
		\
		\begin{center}
			\begin{tabular}{ccccc}
				{} & {}&{} & 1 & {}\\
				0x & 5 & 0 & 3 & c\\
				0x & {}& {}& 0 & 8\\
				+ & {} & {} & {} & {} \\
				\hline
				0x & 5 & 0 & 4 & 4
			\end{tabular}
		\end{center}
		\item
		\
		\begin{center}
			\begin{tabular}{ccccc}
				{} & 4 & F & {F3} & {}\\
				0x & $\cancel{5}$ & $\cancel{0}$ & $\cancel{3}$ & c\\
				0x & {}& {}& 4 & 0\\
				- & {} & {} & {} & {} \\
				\hline
				0x & 4 & F & F & C
			\end{tabular}
		\end{center}
		\item Note that $64_{10}=\texttt{0x40}$:
		\begin{center}
			\begin{tabular}{ccccc}
				0x & 5 & 0 & 3 & c\\
				0x & {}& {}& 4 & 0\\
				+ & {} & {} & {} & {} \\
				\hline
				0x & 5 & 0 & 7 & c
			\end{tabular}
		\end{center}
		\item
		\
		\begin{center}
			\begin{tabular}{ccccc}
				{} & {}&{} & D & FA\\
				0x & 5 & 0 & $\cancel{e}$ & $\cancel{a}$\\
				0x & 5 & 0 & 3 & c\\
				- & {} & {} & {} & {} \\
				\hline
				0x & 0 & 0 & A & E
			\end{tabular}
		\end{center}
	\end{enumerate}

\end{sol}

\begin{ex}{2.5}
	Consider the following three calls to \texttt{show\_bytes}:
\begin{lstlisting}
int val = 0x87654321;
byte_pointer valp = (byte_pointer) &val;
show_bytes(valp, 1);  /* A. */
show_bytes(valp, 2);  /* B. */
show_bytes(valp, 3);  /* C. */
\end{lstlisting}
	Indicate the values that will be printed by each call on a little-endian
	machine and on a big-endian machine.
\end{ex}

\begin{sol}
	Recall \texttt{show\_bytes} function accepts an \texttt{unsigned char*} and
	the size of the data type, which it uses to know how many bytes to read.
	For example, we might pass \texttt{sizeof(int32\_t)}, which would pass
	4 because an \texttt{int32\_t} takes up 32 bits, or 4 bytes. With
	that out of the way:
	\begin{enumerate}[label=(\alph*)]
		\item The call with 1 means to take 1 byte, which on a little-endian will be
		the least significant byte and on big-endian machine will be the most
		significant byte. A byte is 8 bits, or two hexadecimal numbers. Hence,
		the most least significant byte is \texttt{21}, and the most significant is
		\texttt{87}.
		\item On little-endian, it would be \texttt{21} \texttt{43}, and on
		big-endian, it would be \texttt{87} \texttt{65}.
		\item On a little-endian it's \texttt{21} \texttt{43} \texttt{65}, and
		on big-endian it's \texttt{87} \texttt{65} \texttt{43}.
	\end{enumerate}
\end{sol}

\begin{ex}{2.6}
	Using \texttt{show\_int} and \texttt{show\_float}, we determine that the integer
	3510593 has hexadecimal representation \texttt{0x00359141}, while the
	floating-point number 3510593.0 has hexadecimal representation \texttt{0x4A564504}.
	\begin{enumerate}[label=(\alph*)]
		\item Write the binary representations of these two hexadecimal values.
		\item Shift these two strings relative to one another to maximize the number of
		matching bits. How many bits match?
		\item What parts of the strings do not match?
	\end{enumerate}
\end{ex}

\begin{sol}
	\
	\begin{enumerate}
		\item \
		\begin{center}
			\begin{tabular}{c|c}
				Hexadecimal & Binary\\
				\hline
				\texttt{0x00359141} &  0000 0000 0011 0011 0101 0001 0100 0001\\
				\texttt{0x4A564504} & 0100 1010 0101 0110 0100 0101 0000 0100
			\end{tabular}
		\end{center}
		\item The shifted numbers are shown below, with the 21 matching bits shown in
		bold:
		\begin{align*}
					00000000001 &\textbf{101011001000101000001}\\
	        		  010010100 &\textbf{101011001000101000001}00
		\end{align*}
		\item The last two bits in the float do not match. Also, the leading bits
		in integer, $00000000001$, do not match the leading bits in the float:
		$010010100$.
	\end{enumerate}
\end{sol}

\begin{ex}{2.7}
	What would be printed as a result of the following call to \texttt{show\_bytes}?
	\begin{lstlisting}
const char *s = "abcdef";
show_bytes((byte_pointer) s, strlen(s));
	\end{lstlisting}
	Note that letters `\texttt{a}' through `\texttt{z}' have ASCII codes \texttt{0x61}
	through \texttt{0x7A}.
\end{ex}

\begin{sol}
	The output for the lowercase characters would be: \texttt{0x61 0x62 0x63 0x64 0x65 0x66 0x00}
	on any system using ASCII as its character code. The \texttt{0x00} is the null character
	used to terminate strings in C.
\end{sol}

\begin{ex}{2.8}
	Fill in the following table showing the results of evaluating Boolean operations on
	bit vectors.
	\begin{center}
		\begin{tabular}{cc}
			Operationg & Result\\
			\hline
			$a$ & [01101001]\\
			$b$ & [01010101]\\
			$\sim a$ & \makebox[1cm]{\hrulefill} \\
			$\sim b$ & \makebox[1cm]{\hrulefill} \\
			$a\, \& \,b$ & \makebox[1cm]{\hrulefill} \\
			$a \, | \, b$ & \makebox[1cm]{\hrulefill} \\
			$a ^\wedge b $ & \makebox[1cm]{\hrulefill} 
		\end{tabular}
	\end{center}
\end{ex}

\begin{sol}
	Treating the bit sequences as bit vectors and noting that $\sim$ is logical NOT,
	$\&$ is logical AND, $|$ is logical OR, and $^\wedge$ is logical XOR, we get:
	\begin{center}
		\begin{tabular}{cc}
			Operationg & Result\\
			\hline
			$a$ & [01101001]\\
			$b$ & [01010101]\\
			$\sim a$ & [10010110] \\
			$\sim b$ & [10101010] \\
			$a\, \& \,b$ & [01000001] \\
			$a \, | \, b$ & [01111101] \\
			$a ^\wedge b $ & [00111100]
		\end{tabular}
	\end{center}
\end{sol}

\begin{ex}{2.9}
	Computers generate color pictures on a video screen or liquid crystal display by
	mixing three different colors of light: red, green, and blue. Imagine a simple
	scheme, with three different lights, each of which can be turned on or off,
	projecting onto a glass screen.
	\
	We can then create eight different colors based on the absence (0) or presence (1)
	of light sources $R$, $G$, and $B$:
	\begin{center}
		\begin{tabular}{cccc}
			$R$ & $G$ & $B$ & Color\\
			\hline
			0 & 0 & 0 & Black\\
			0 & 0 & 1 & Blue \\
			0 & 1 & 0 & Green\\
			0 & 1 & 1 & Cyan \\
			1 & 0 & 0 & Red \\
			1 & 0 & 1 & Magenta\\
			1 & 1 & 0 & Yellow\\
			1 & 1 & 1 & White
		\end{tabular}
	\end{center}
	Each of these colors can be represented as a bit vector of length 3, and we can
	apply Boolean operations to them.
	\begin{enumerate}[label=(\alph*)]
		\item The complement of a color is formed by turning off the lights that are
		on and turning on the lights that are off. What would be the complement of
		each of the eight colors listed above?
		\item Describe the effect of applying Boolean operations on the following
		colors:
		\begin{center}
			Blue $|$ Green = \makebox[1cm]{\hrulefill} \\
			Yellow \& Cyan = \makebox[1cm]{\hrulefill} \\
			Red $^\wedge$ Magenta = \makebox[1cm]{\hrulefill} 
		\end{center}
	\end{enumerate}
\end{ex}

\begin{sol}
	\begin{enumerate}[label=(\alph*)]
		\item The augmented table below shows the complementary colors, obtained by applying
		the logical NOT operation $\sim$ to each bit vector:
		\begin{center}
			\begin{tabular}{ccccc}
				$R$ & $G$ & $B$ & Color & Complement\\
				\hline
				0 & 0 & 0 & Black & White\\
				0 & 0 & 1 & Blue & Yellow \\
				0 & 1 & 0 & Green & Magenta\\
				0 & 1 & 1 & Cyan  & Red  \\
				1 & 0 & 0 & Red & Cyan \\
				1 & 0 & 1 & Magenta & Green\\
				1 & 1 & 0 & Yellow & Blue\\
				1 & 1 & 1 & White & Black
			\end{tabular}
		\end{center}
		\item Blue $|$ Green means we apply the logical OR operation to their corresponding bit
		vectors. We get 001 | 010 = 011, which is Cyan. Yellow \& Cyan means we apply the logical
		AND operator to the bit vectors, so 110 \& 011 = 010, which is Green. Finally,
		Red $^\wedge$ Magenta means apply logical XOR to the bit vectors, so 100 $^\wedge$ 101 = 001,
		which is Blue, so
		\begin{center}
			Blue $|$ Green = Cyan \\
			Yellow \& Cyan = Green \\
			Red $^\wedge$ Magenta = Blue
		\end{center}
	\end{enumerate}
\end{sol}

\begin{ex}{10}
	As an application of the property that $a ^\wedge a=0$ for any bit vector $a$, consider
	the following program:
	\begin{lstlisting}
void inplace_swap(int *x, int *y) {
   	*y = *x ^ *y;    /* Step 1 */
   	*x = *x ^ *y;    /* Step 2 */
   	*y = *x ^ *y;    /* Step 3 */
}
	\end{lstlisting}
	As the name suggests, we claim that the effect of this procedure is to swap the values
	stored at the locations denoted by pointer variables \texttt{x} and \texttt{y}. Note
	that unlike the usual technique for swapping two values, we do not need a third location
	to temporarily store one value while we are moving the other. There is no performance
	advantage to this way of swapping; it is merely an intellectual amusement.
	\
	
	\noindent Staring with values $a$ and $b$ in the locations pointed to by \texttt{x} and
	\texttt{y}, respectively, fill int he table that follows, giving the values stored at the
	two locations after each step of the procedure. Use the properties of $^\wedge$ to
	show that the desired effect is achieved. Recall that every element is its own additive
	inverse (that is, $a^\wedge a=0$).
	\begin{center}
		\begin{tabular}{ccc}
			Step & \texttt{*x} & \texttt{*y}\\
			\hline
			Initially & $a$ & $b$ \\
			Step 1 & \makebox[1cm]{\hrulefill}  & \makebox[1cm]{\hrulefill} \\
			Step 2 & \makebox[1cm]{\hrulefill}  & \makebox[1cm]{\hrulefill} \\
			Step 3 & \makebox[1cm]{\hrulefill}  & \makebox[1cm]{\hrulefill}
		\end{tabular}
	\end{center}
\end{ex}

\begin{sol}
	The completed table is below:
	
	\begin{center}
		\begin{tabular}{ccc}
			Step & \texttt{*x} & \texttt{*y}\\
			\hline
			Initially & $a$ & $b$ \\
			Step 1 & $a$  & $a ^\wedge b$ \\
			Step 2 & $a ^\wedge (a^\wedge b)$ & $a ^\wedge b$ \\
			Step 3 &  $a ^\wedge (a^\wedge b)$  & $[a ^\wedge (a^\wedge b)] ^\wedge [a ^\wedge b]$
		\end{tabular}
	\end{center}
	Since $a ^\wedge (a^\wedge b)=(a^\wedge a)\wedge b=0^\wedge b=b$, the table evaluates
	correctly.
\end{sol}

\begin{ex}{2.11}
	Armed with the function \texttt{inplace\_swap} from Problem 2.10, you decide to write
	code that will reverse the elements of an array by swapping from opposite ends of the
	array, working toward the middle:
	\begin{lstlisting}
void reverse_array(int a[], int cnt) {
	int first, last;
	for (first = 0, last = cnt-1;
	     first <= last;
	     first++, last--)
	     inplace_swap(&a[first], &a[last]);
}
	\end{lstlisting}
	When you apply your function to an array containing elements $1$, $2$, $3$, and $4$,
	you find that the array now has, as expected, elements $4$, $3$, $2$, and $1$. When
	you try it on an array with elements $1$, $2$, $3$, $4$, and $5$, however, you are
	surprised to see that the array now has elements $5$, $4$, $0$, $2$, and $1$. In
	fact, you discover that the code always works on arrays of even length, but it sets
	the middle element to 0 whenever the array has odd length.
	\begin{enumerate}[label=(\alph*)]
		\item For an array of odd length $\texttt{cnt}=2k+1$, what are the values of
		variables $\texttt{first}$ and $\texttt{last}$ in the final iteration of
		function \texttt{reverse\_array}?
		\item Why does this call to function \texttt{inplace\_swap} set the array element
		to 0?
		\item What simple modification to the code for \texttt{reverse\_array} would
		eliminate this problem?
	\end{enumerate}
\end{ex}

\begin{sol}
	\
	\begin{enumerate}[label=(\alph*)]
		\item Their values are the same, and their value is the one at
		the center of the array, namely, \texttt{a[cnt / 2]}.
		\item The XOR operation operates on the same number, and
		since every element is its own additive inverse with respect to
		this operation, the result is 0.
		\item Replace the comparison \texttt{first <= last} with
		\texttt{first < last}.
	\end{enumerate}
\end{sol}

\begin{ex}{2.12}
	Write C expressions, in terms of variable \texttt{x}, for the following values.
	Your code should work for any size $w\geq 8$. For reference, we show the result
	of evaluating the the expressions for \texttt{x = 0x87654321}, with $w=32$.
	\begin{enumerate}[label=(\alph*)]
		\item The least significant byte of \texttt{x}, with all other bits set
		to $0$. [\texttt{0x00000021}].
		\item All but the least significant byte of \texttt{x} complemented, with
		the least significant byte left unchanged. [\texttt{0x789ABC21}]
		\item The least significant byte set to all ones, and all other bytes
		of \texttt{x} left unchanged. [\texttt{0x876543FF}]
	\end{enumerate}
\end{ex}

\begin{sol}
	\
	\begin{enumerate}[label=(\alph*)]
		\item \texttt{x \& 0xFF}
		\item  The expression is: \texttt{(x \& 0xFF) | (\~{}x \& \~{}0xFF)}. First,
		we use the \texttt{0xFF} mask to get the first byte of $x$. Then, we complement
		\texttt{x}, but mask with \texttt{\~{}0xFF} instead to get all bits except
		the last byte.
		\item The expression is: \texttt{x |{}0xFF}. By using the logical OR, we
		ensure the least significant byte is set to all 1s. The upper bytes are
		0, so they do not change. what's in \texttt{x}.
	\end{enumerate}
\end{sol}

\begin{ex}{2.13}
	The Digital Equipment VAX computer was a very popular machine from the late 1970s until
	the late 1980s. Rather than instructions for Boolean operations AND and OR, it had
	instructions \texttt{bis} (bit set) and \texttt{bic} (bit clear). Both instructions
	take a data word \texttt{x} and a mask word \texttt{m}. They generate a result
	\texttt{z} consisting of the bits of \texttt{x} modified according to the bits of
	\texttt{m}. With \texttt{bis}, the modification involves setting \texttt{z} to 1
	at each position where \texttt{m} is 1. With \texttt{bic}, the modification involves
	setting \texttt{z} to 0 at each bit position where \texttt{m} is 1.
	
	\
	To see how these operations relate to C bit-level operations, assume we have functions
	\texttt{bis} and \texttt{bis} implementing the bit set and bit clear operations, and
	that we want to use these to implement functions computing bitwise operations \texttt{|}
	and \texttt{\^}. Fill in the missing code below. Write C expressions for the operations
	\texttt{bis} and \texttt{bic}.
	
	\begin{lstlisting}
/* Declarations of functions implementing operatings bis and bic */
int bis(int x, int m);
int bic(int x, int m);

/* Compute x|y using only calls to fucntions bis and bic */
int bool_or(int x, int y) {
	int result = __________;
	return result;
}

/* Compute x^y using only calls to functions bis and bic */
int bool_xor(int x, int y) {
	int result = ___________;
	return result;
}
	\end{lstlisting}
\end{ex}

\begin{sol}
	For the OR operation, suppose we start with $x$. The expression \texttt{bis(x, 0)} is
	equivalent to \texttt{x}. This is because 0 does not have any 1 bits. If \texttt{x} has
	1 bits, they remain 1; if they're 0, they remain 0. On the other hand, if
	\texttt{y} had all 1 bits, then \texttt{bit(x, y)} will be all 1 bits, regardless of what
	was in \texttt{x}. This suggests the correct way to implement \texttt{x} OR \texttt{y}
	is with \texttt{bis(x,y)}:
	\begin{enumerate}
		\item If the $i$-th bit of \texttt{x} is 1, then the result is 1 regardless
		of the value of the $i$-th bit of \texttt{y}.
		\item  If the $i$-th bit of \texttt{x} is 0, then the result is only 1 if the
		$i$-th bit of \texttt{y} is 1.
	\end{enumerate}
	Consider a truth table for the \texttt{bic(x, y)} operation:
	\begin{center}
		\begin{tabular}{cc|c}
			\texttt{x} & \texttt{y} & \texttt{bic(x, y)}\\
			\hline
			0 & 0 & 0 \\
			0 & 1 & 0 \\
			1 & 0 & 1 \\
			1 & 1 & 0 
		\end{tabular}
	\end{center}
	Note that if \texttt{x} is 0, then the result is 0. We could flip the inputs and get:
	\begin{center}
		\begin{tabular}{cc|c}
			\texttt{y} & \texttt{x} & \texttt{bic(y, x)}\\
			\hline
			0 & 0 & 0 \\
			0 & 1 & 0 \\
			1 & 0 & 1 \\
			1 & 1 & 0 
		\end{tabular}
	\end{center}
	Therefore, the $i$-th bit of \texttt{bic(x, y)} will be 1 only if the
	$i$-th bit of \texttt{x} is 1 and the $i$-th bit of \texttt{y} is 0.
	The opposite is true for \texttt{bic(y, x)}. We therefore get \texttt{x \^{} y}
	by applying the OR operation (which is just \texttt{bis}):
	\begin{lstlisting}
/* Declarations of functions implementing operatings bis and bic */
int bis(int x, int m);
int bic(int x, int m);

/* Compute x|y using only calls to fucntions bis and bic */
int bool_or(int x, int y) {
	int result = bis(x, y);
	return result;
}

/* Compute x^y using only calls to functions bis and bic */
int bool_xor(int x, int y) {
	int result = bis(bic(x, y), bic(y, x));
	return result;
}
	\end{lstlisting}
\end{sol}

\begin{ex}{2.14}
	Suppose that \texttt{x} and \texttt{y} have byte values \texttt{0x66} and
	\texttt{0x39}, respectively. Fill in the following table indicating the byte
	values of the different C expressions:
	\begin{center}
		\begin{tabular}{cc|cc}
			Expression & Value & Expression & Value\\
			\hline
			\texttt{x \& y} & \makebox[1cm]{\hrulefill} & \texttt{x \&\& y} & \makebox[1cm]{\hrulefill}\\
			\texttt{x | y} & \makebox[1cm]{\hrulefill} & \texttt{x || y} & \makebox[1cm]{\hrulefill}\\
			\texttt{\~{} x | \~{}y} & \makebox[1cm]{\hrulefill} & \texttt{!x || !y} & \makebox[1cm]{\hrulefill}\\
			\texttt{x \& !y} & \makebox[1cm]{\hrulefill} & \texttt{x \&\& \~{}y} & \makebox[1cm]{\hrulefill}\\
		\end{tabular}
	\end{center}
\end{ex}

\begin{sol}
	Note that \texttt{x = 0x66 = 0110 0110} and \texttt{y = 0x39 = 0011 1001}. Also
	\texttt{\~{}x = 1001 1001} and \texttt{\~{}y = 1100 0110}.
	\begin{center}
	
		\begin{tabular}{cc|cc}
			Expression & Value & Expression & Value\\
			\hline
			\texttt{x \& y} & \texttt{0010 0000} & \texttt{x \&\& y} & \texttt{0x01}\\
			\texttt{x | y} & \texttt{0111 1111} & \texttt{x || y} & \texttt{0x01}\\
			\texttt{\~{}x | \~{}y} & \texttt{1101 1111} & \texttt{!x || !y} & \texttt{0x00}\\
			\texttt{x \& !y} & \texttt{0100 0110} & \texttt{x \&\& \~{}y} & \texttt{0x01}\\
		\end{tabular}
	\end{center}
\end{sol}

\begin{ex}{2.15}
	Using only bit-level and logical operations, write a C expression that is equivalent
	to \texttt{x == y}. That is, it will return 1 when \texttt{x} and \texttt{y} are equal
	and 0 otherwise.
\end{ex}

\begin{sol}
	We can use \texttt{!(x \& \~{}y)}. Suppose \texttt{x} and \texttt{y} are the same;
	then \texttt{x \& \~{}y} is 0, so the logical NOT operator \texttt{!} makes the result
	1. Now suppose they're different, say, they're $i$-th bit is different. Then
	the $i$-th bit of \texttt{x} \& \texttt{\~{}y} is the same, so \texttt{x \& \~{}y} is
	nonzero, which means that applying logical NOT gives a value of 0.
\end{sol}

\begin{ex}{2.16}
	Fill in the table below showing the effects of the different shift operations on
	single-byte quantities. The best way to think about shift operations is to work
	with binary representations. Convert the initial values to binary, perform the
	shifts, and then convert back to hexadecimal. Each of the answers should be 8
	binary digits or 2 hexadecimal digits.
	\begin{center}
		\begin{tabular}{c|cccc}
			\texttt{x} & \texttt{0xC3} & \texttt{0x75} & \texttt{0x87} & \texttt{0x66} \\
			\hline
			\texttt{x << 3} \\
			\texttt{x >> 2} (Logical) \\
			\texttt{x >> 2} (Arithmetic)
		\end{tabular}
	\end{center}
\end{ex}

\begin{sol}
	We begin by converting each hexadecimal to binary:
	\begin{align*}
		\texttt{0xC3} &= \texttt{1100 0011}\\
		\texttt{0x75} &= \texttt{0111 0101}\\
		\texttt{0x87} &= \texttt{1000 0111}\\
		\texttt{0x66} &= \texttt{0110 0110}
	\end{align*}
	From here, shifts are easy:
	\begin{center}
		\begin{tabular}{c|cccc}
			\texttt{x} (Hex) & \texttt{0xC3} & \texttt{0x75} & \texttt{0x87} & \texttt{0x66} \\
			\texttt{x} (Binary) & \texttt{1100 0011} & \texttt{0111 0101} & \texttt{1000 0111} & \texttt{0110 0110} \\
			\hline
			\texttt{x << 3} (Hex) & \texttt{0x18} & \texttt{0xA8} & \texttt{0x38} &  \texttt{0x30}\\
			\texttt{x << 3} (Binary) & \texttt{0001 1000} & \texttt{1010 1000} & \texttt{0011 1000} &  \texttt{0011 0000}\\
			\hline
			
			\texttt{x >> 2} (Logical, Hex) & \texttt{0x30} & \texttt{0x1D} & \texttt{0x21} & \texttt{0x19} \\
			\texttt{x >> 2} (Logical, Binary) & \texttt{0011 0000} & \texttt{0001 1101} & \texttt{0010 0001} & \texttt{0001 1001} \\
			\hline
			
			\texttt{x >> 2} (Arithmetic, Hex)  & \texttt{0xF0} & \texttt{0x1D} & \texttt{0xE1} & \texttt{0x19} \\
			\texttt{x >> 2} (Arithmetic, Binary)  & \texttt{1111 0000} & \texttt{0001 1101} & \texttt{1110 0001} & \texttt{0001 1001} 
		\end{tabular}
	\end{center}
\end{sol}

\begin{ex}{2.17}
	Assuming $w=4$, we can assign a numeric value to each possible hexadecimal digit,
	assuming either an unsigned or a two's-complement interpretation. Fill in the following
	table according to these interpretations by writing out the nonzero powers of $2$
	in the summations shown in Equations~\ref{eq:b2u} and~\ref{eq:b2t}:
	\begin{equation}\label{eq:b2u}
		B2U_w(\vec{x})\doteq\sum_{i=0}^{w-1}x_i2^i
	\end{equation}
	\begin{equation}\label{eq:b2t}
		B2T_w(\vec{x})\doteq-x_{w-1}2^{w-1}+\sum_{i=0}^{w-2}x_i2^i
	\end{equation}
	where $w$ is a positive integer, $x_i\in\{0,1\}$, and $\vec{x}=[x_{w-1},x_{w-2},\ldots,x_0]$.
	\begin{center}
		\begin{tabular}{cccc}
			Hexadecimal & Binary & $B2U_4(\vec{x})$ & $B2T_4(\vec{x})$ \\
			\hline
			\texttt{0xE} & \texttt{[1110]} & $2^3+2^2+2^1=14$ & $-2^3+2^2+2^1=-2$\\
			\texttt{0x0} & \makebox[1cm]{\hrulefill} & \makebox[1cm]{\hrulefill} & \makebox[1cm]{\hrulefill}\\
			\texttt{0x5} & \makebox[1cm]{\hrulefill} & \makebox[1cm]{\hrulefill} & \makebox[1cm]{\hrulefill}\\
			\texttt{0x8} & \makebox[1cm]{\hrulefill} & \makebox[1cm]{\hrulefill} & \makebox[1cm]{\hrulefill}\\
			\texttt{0xD} & \makebox[1cm]{\hrulefill} & \makebox[1cm]{\hrulefill} & \makebox[1cm]{\hrulefill}\\
			\texttt{0xF} & \makebox[1cm]{\hrulefill} & \makebox[1cm]{\hrulefill} & \makebox[1cm]{\hrulefill}
		\end{tabular}
	\end{center}
\end{ex}

\begin{sol}
	\
	\begin{center}
		\begin{tabular}{cccc}
			Hexadecimal & Binary & $B2U_4(\vec{x})$ & $B2T_4(\vec{x})$ \\
			\hline
			\texttt{0xE} & \texttt{[1110]} & $2^3+2^2+2^1=14$ & $-2^3+2^2+2^1=-2$\\
			\texttt{0x0} & \texttt{[0000]} & $0$ & $0$\\
			\texttt{0x5} & \texttt{[0101]} & $2^2+2^0=5$ & $2^2+2^0=5$\\
			\texttt{0x8} & \texttt{[1000]} & $2^3=8$ & $-2^{-3}=-8$\\
			\texttt{0xD} & \texttt{[1101]} & $2^3+2^2+2^0=13$ & $-2^3+2^2+2^0=-3$\\
			\texttt{0xF} & \texttt{[1111]} & $2^3+2^2+2^1+2^0=15$ & $-2^{-3}+2^2+2^1+2^0=-1$
		\end{tabular}
	\end{center}
\end{sol}

\begin{ex}{2.18}
	In Chapter 3, we will look at listings generated by a \emph{diassembler}, a program that
	converts an executable program file back to a more readable ASCII form. These
	files contain many hexadecimal numbers, typically representing values in two's-complement
	form. Being able to recognize these numbers and understand their significance
	(for example, whether they are negative or positive) is an important skill.
	
	\
	\noindent For lines labeled A-I (on the right) in the following listing, convert the
	hexadecimal values (in 32-bit two's complement form) shown to the right of the
	instruction names (\texttt{sub}, \texttt{mov}, and \texttt{add}) into their decimal
	equivalents:
	\begin{lstlisting}{}
4004d0:   48 81 ec e0 02 00 00     sub    $0x2e0,%rsp                 A.
4004d7:   48 8b 44 24 a8           mov    -0x58(%rsp),%rax            B.
4004dc:   48 03 47 28              add    0x28(%rdi),%rax             C.
4004e0:   48 89 44 24 d0           mov    %rax,-0x30(%rsp)            D.
4004e5:   48 8b 44 24 78           mov    %0x78(%rsp),%rax            E.
4004ea:   48 89 87 88 00 00 00     mov    %rax,0x88(%rdi)             F.
4004f1:   48 8b 84 24 f8 01 00     mov    0x1f8(%rsp),%rax            G.
4004f8:   00
4004f9:   48 03 44 24 08           add    0x8(%rsp),%rax
4004fe:   48 89 84 24 c0 00 00     mov    %rax,%0xc0(%rsp)            H.
400505:   00
400506:   48 8b 44 d4 b8           mov    -0x48(%rsp,%rdx,8),%rax     I.
	\end{lstlisting}
\end{ex}

\begin{sol}
	\
	\begin{enumerate}[label=(\Alph*)]
		\item We are given \texttt{\texttt{$\bar{x}$} = 0x2e0 = 0010 1110 0000}. Since the number
		is given in 32-bit two's complement form, the other 20 bits are 0. By using the $B2T_{32}$
		function, we get
		\begin{align*}
			B2T_{32}(\bar{x})=2^{9}+2^{7}+2^{6}+2^{5}=736
		\end{align*}
		\item Letting \texttt{$\bar{x}$ = 0x58 = 0101 1000}, we interpret the \texttt{-} as
		a negative sign and get
		\begin{align*}
			-B2T_{32}(\bar{x})=-\left(2^{6} + 2^{4} + 2^{3}=2,147,483,560\right)=-88
		\end{align*}
		\item Since \texttt{$\bar{x}$ = 0x28 = 0010 1000}, we get
		\begin{align*}
			B2T_{32}(\bar{x})=2^5+2^3=40
		\end{align*}
		\item Letting \texttt{$\bar{x}$ = 0x30 = 0011 0000}, we get
		\begin{align*}
			-B2T_{32}(\bar{x})=-\left(2^5+2^4\right) = -48
		\end{align*}
		\item Letting \texttt{$\bar{x}$ = 0x78 = 0111 1000}, we get
		\begin{align*}
			B2T_{32}(\bar{x})=2^6+2^5+2^4+2^3=120
		\end{align*}
		\item Letting \texttt{$\bar{x}$ = 0x88 = 1000 1000}, we get
		\begin{align*}
			B2T_{32}(\bar{x}) = 2^7 + 2^3 = 136
		\end{align*}
		\item Letting \texttt{$\bar{x}$ = 0x1f8 = 0001 1111 1000}, we get
		\begin{align*}
			B2T_{32}(\bar{x})=2^8 + 2^7 + 2^6 + 2^5 + 2^4 + 2^3 = 504
		\end{align*}
		\item Letting \texttt{$\bar{x}$ = 0xc0 = 1100 0000}, we get
		\begin{align*}
			B2T_{32}(\bar{x})=2^{7}+2^{6}=192
		\end{align*}
		\item Letting \texttt{$\bar{x}$ = 0x48 = 0100 1000}, we get
		\begin{align*}
			-B2T_{32}(\bar{x}) = -\left(2^6 + 2^3\right) = 72
		\end{align*}
	\end{enumerate}
\end{sol}

\begin{ex}{2.19}
	Using the table you filled when solving Problem 2.17, fill in the following table describing
	the function $T2U_4$:
	\begin{center}
		\begin{tabular}{cc}
			$x$ & $T2U_4(x)$\\
			\hline
			-8 & \makebox[1cm]{\hrulefill}\\
			-3 & \makebox[1cm]{\hrulefill}\\
			-2 & \makebox[1cm]{\hrulefill}\\
			-1 & \makebox[1cm]{\hrulefill}\\
			 0 & \makebox[1cm]{\hrulefill}\\
			 5 & \makebox[1cm]{\hrulefill}
		\end{tabular}
	\end{center}
\end{ex}

\begin{sol}
	\
	\begin{center}
		\begin{tabular}{cc}
			$x$ & $T2U_4(x)$\\
			\hline
			-8 & 8\\
			-3 & 13\\
			-2 & 14\\
			-1 & 15\\
			0 & 0\\
			5 & 5
		\end{tabular}
	\end{center}
\end{sol}

\begin{ex}{2.20}
	Explain how Equation~\ref{eqn:t2u} applies to the entries in the table you generated when
	solving Problem 2.19.
	\begin{equation}\label{eqn:t2u}
		T2U_w(x)=\begin{cases}
			x + 2^w, & x < 0\\
			x,       & x\geq 0
		\end{cases}
	\end{equation}
\end{ex}

\begin{sol}
	In Problem 2.19, we had $w=4$. If the value on the left is non-negative, it remains unchanged.
	Otherwise, we add $2^4=16$ to the corresponding value on the left column.
\end{sol}

\begin{ex}{2.21}
	Assuming the expressions are evaluated when executing a 32-bit program on a machine that uses
	two's-complement arithmetic, fill in the following table describing the effect of casting
	and relational operations, in the style of Figure 2.19 (of text).
		\begin{center}
		\begin{tabular}{ccc}
			\hline
			Expression & Type & Evaluating \\
			\hline
			\texttt{-2147483647-1 == 2147483648U} & \makebox[1cm]{\hrulefill} & \makebox[1cm]{\hrulefill}\\
			\texttt{-2147483647-1 == 2147483647} & \makebox[1cm]{\hrulefill} & \makebox[1cm]{\hrulefill}\\
			\texttt{-2147483647-1U == 2147483647} & \makebox[1cm]{\hrulefill} & \makebox[1cm]{\hrulefill}\\
			\texttt{-2147483647-1 == -2147483647} & \makebox[1cm]{\hrulefill} & \makebox[1cm]{\hrulefill}\\
			\texttt{-2147483647-1U == -2147483647} & \makebox[1cm]{\hrulefill} & \makebox[1cm]{\hrulefill}\\
		\end{tabular}
	\end{center}
\end{ex}

\begin{sol}
	\
	\begin{center}
		Note that in the operation \texttt{-2147483647-1U}, the operand \texttt{1U} is unsigned,
		so C implicitly casts \texttt{-2147483647} to the unsigned number $2147483647+2^{32}=2147483649$.
		\begin{tabular}{ccc}
			\hline
			Expression & Type & Evaluating \\
			\hline
			\texttt{-2147483647-1 == 2147483648U} & Unsigned & \texttt{1}\\
			\texttt{-2147483647-1 == 2147483647} & Signed & \texttt{0}\\
			\texttt{-2147483647-1U == 2147483647} & Unsigned & \texttt{0} \\
			\texttt{-2147483647-1 == -2147483647} & Signed & \texttt{1}\\
			\texttt{-2147483647-1U == -2147483647} & Unsigned & \texttt{0}\\
		\end{tabular}
	\end{center}
\end{sol}

\begin{ex}{2.22}
	Show that each of the following bit vectors is a two's-complement representation of -5
	by applying Equation~\ref{eq:b2t}:
	\begin{enumerate}[label=(\alph*)]
		\item {}[1011]
		\item {}[11011]
		\item {}[111011]
	\end{enumerate}
	Observe that the second and third bit vectors can be derived from the first by sign extension.
\end{ex}

\begin{sol}
	\
	\begin{enumerate}[label=(\alph*)]
		\item $w=4$, so $B2T_4([1011])=-2^3+2^1+2^0=-8+2+1=-5$.
		\item $w=5$, so $B2T_5([11011])=-2^4+2^3+2^1+2^0=-16+8+2+1=-5$.
		\item $w=6$, so $B2T_6([111011])=-2^5+2^4+2^3+2^1+2^0=-32+16+8+2+1=-5$.
	\end{enumerate}
\end{sol}

\begin{ex}2{2.23}
	Consider the following C functions:
	\begin{lstlisting}
int fun1(unsigned word) {
	return (int) ((word << 24) >> 24);
}

int fun2(unsigne dword) {
	return ((int) word << 24) >> 24;
}
	\end{lstlisting}
	Assume these are executed as a 32-bit program on a machine that uses two's-complement arithmetic.
	Assume also that right shifts og signed values are performed arithmetically, while right shifts
	of unsigned values are performed logically.
	\begin{enumerate}[label=(\alph*)]
		\item Fill in the following table showing the effect of these functions for several example
		arguments. You will find it more convenient to work with a hexadecimal representation. Just
		remember that hex digits \texttt{8} through \texttt{F} have their most significant bits
		equal to 1.
		\begin{center}
			\begin{tabular}{ccc}
				\texttt{w} & \texttt{fun1(w)} & \texttt{fun2(w)}\\
				\hline
				\texttt{0x00000076} & \makebox[1cm]{\hrulefill} & \makebox[1cm]{\hrulefill}\\
				\texttt{0x87654321} & \makebox[1cm]{\hrulefill} & \makebox[1cm]{\hrulefill}\\
				\texttt{0x000000C9} & \makebox[1cm]{\hrulefill} & \makebox[1cm]{\hrulefill}\\
				\texttt{0xEDCBA987} & \makebox[1cm]{\hrulefill} & \makebox[1cm]{\hrulefill}
			\end{tabular}
		\end{center}
		\item Describe in words the useful computation each of these functions perform.
	\end{enumerate}
\end{ex}

\begin{sol}
	\
	\begin{enumerate}[label=(\alph*)]
		\item
		\
		\begin{center}
			\begin{tabular}{ccc}
				\texttt{w} & \texttt{fun1(w)} & \texttt{fun2(w)}\\
				\hline
				\texttt{0x00000076} & \texttt{0x00000076} & \texttt{0x00000076}\\
				\texttt{0x87654321} & \texttt{0x00000021} & \texttt{0x00000021}\\
				\texttt{0x000000C9} & \texttt{0x000000C9} & \texttt{0xFFFFFFC9}\\
				\texttt{0xEDCBA987} & \texttt{0x00000087} & \texttt{0xFFFFFF87}
			\end{tabular}
		\end{center}
		\item \texttt{fun1} computes the zero-extension of the least significant byte,
		whereas \texttt{fun2} computes the sign-extension of the least significant byte.
		and the 
	\end{enumerate}
\end{sol}

\begin{ex}{2.24}
	Suppose we truncate a 4-bit value (represented by hex digits \texttt{0} through
	\texttt{F}) to a 3-bit value (represented as hex digits \texttt{0} through \texttt{7}).
	Fill in the table below showing the effect of this truncation for some cases, in terms
	of the unsigned two's-complement interpretation of those bit patterns.
	\begin{center}
		\begin{tabular}{cccccc}
			\multicolumn{2}{c}{Hex} & \multicolumn{2}{c}{Unsigned} & \multicolumn{2}{c}{Two's Complement}\\
			Original & Truncated & Original & Truncated & Original & Truncated\\
			\hline
			0 & 0 & 0 & \makebox[1cm]{\hrulefill} & 0 & \makebox[1cm]{\hrulefill}\\
			2 & 2 & 2 & \makebox[1cm]{\hrulefill} & 2 & \makebox[1cm]{\hrulefill}\\
			9 & 1 & 9 & \makebox[1cm]{\hrulefill} & -7 & \makebox[1cm]{\hrulefill}\\
			B & 3 & 11 & \makebox[1cm]{\hrulefill} & -5 & \makebox[1cm]{\hrulefill}\\
			F & 7 & 15 & \makebox[1cm]{\hrulefill} & -1 & \makebox[1cm]{\hrulefill}
		\end{tabular}
	\end{center}
	Explain how Equations~\ref{eqn:trunc-b2u-k} and ~\ref{eqn:trunc-b2t-k} apply to these cases.
	\begin{equation}\label{eqn:trunc-b2u-k}
		B2U_k([x_{k-1}, x_{k-2},\ldots,x_0])=B2U_w([x_{w-1},x_{w-2},\ldots,x_0])\mod 2^k
	\end{equation}
	\begin{equation}\label{eqn:trunc-b2t-k}
		B2T_k([x_{k-1}, x_{k-2},\ldots,x_0])=U2T_k)(B2U_w([x_{w-1},x_{w-2},\ldots,x_0])\mod 2^k)
	\end{equation}
\end{ex}

\begin{sol}
	\
	\begin{center}
		\begin{tabular}{cccccc}
			\multicolumn{2}{c}{Hex} & \multicolumn{2}{c}{Unsigned} & \multicolumn{2}{c}{Two's Complement}\\
			Original & Truncated & Original & Truncated & Original & Truncated\\
			\hline
			0 & 0 & 0 & 0 & 0 & 0\\
			2 & 2 & 2 & 2 & 2 & 2\\
			9 & 1 & 9 & 1 & -7 & 1\\
			B & 3 & 11 & 3 & -5 & 3\\
			F & 7 & 15 & 7 & -1 & -1
		\end{tabular}
	\end{center}
	The value under the Unsigned  Truncated column is obtained by applying equation~\ref{eqn:trunc-b2u-k},
	where we take apply $x\mod (2^3)=x\mod8$ to all of the values. The value under the Two's Complement
	Truncated column ar obtained by mapping applying equation~\ref{eqn:trunc-b2t-k}, which means
	we just apply $U2T_3$ to the result of the Unsigned Truncated column.
\end{sol}

\begin{ex}{2.25}
	Consider the following code that attempts to sum the elements of an array \texttt{a},
	where the number of elements is given by parameter \texttt{length}:
	\begin{lstlisting}
/* WARNING: This is buggy code */
float sum_elements(float a[], unsigned length) {
	int i;
	float result = 0;
	
	for (i = 0; i <= length-1; i++)
	    result += a[i];
	return result;
}
	\end{lstlisting}
	When run with argument \texttt{length} equal to 0, this code should return \texttt{0.0}.
	Instead, it encounters a memory error. Explain why this happens. Show how this can be
	corrected.
\end{ex}

\begin{sol}
	The argument \texttt{length} is unsigned, so in the operation \texttt{length-1} of
	the loop condition, the operation is equivalent to \texttt{length-1U} because C implicit
	casts the signed operand \texttt{-1} to unsigned. The result is \texttt{0+UINT\_MAX},
	where \texttt{UINT\_MAX} is the constant declared in \texttt{<limits.h>} that represents
	the maximum unsigned number that can be represented on the existing machine.
	When the unsigned comparison then happens, the loop condition is always true because
	no unsigned number exceeds this one. When the index \texttt{i} is then used to index
	into the array \texttt{a} inside the loop, an invalid memory location is accessed,
	causing the error. We can correct the by changing the condition to from
	\texttt{i <= length -1} to \texttt{i < length}.
\end{sol}

\begin{ex}{2.26}
	You are given the assignment of writing a function that determines whether one string is
	longer than another. You decide to make use of the string library function \texttt{strlen}
	having the following declaration:
	\begin{lstlisting}
/* Prototype for library function strlen */
size_t strlen(const char *s);
	\end{lstlisting}
	Here is your first attempt at the function:
	\begin{lstlisting}
/* Determine whether string s is longer than string t */
/* WARNING: This function is buggy */
int strlonger(char *s, char *t) {
	return strlen(s) - strlen(t) > 0;
}
	\end{lstlisting}
	When you test this on some simple data, things do not seem to work quite right. You
	investigate further and determine that, whe compiled as a 32-bit program, data type
	\texttt{size\_t} is defined (via \texttt{typedef}) in header file \texttt{stdio.h} to
	be \texttt{unsigned}.
	\begin{enumerate}[label=(\alph*)]
		\item For what cases will this function produce an incorrect result?
		\item Explain how this incorrect result comes about.
		\item Show how to fix the code so that it will work reliably.
	\end{enumerate}
\end{ex}

\begin{sol}
	\
	\begin{enumerate}[label=(\alph*)]
		\item The function will fail anytime the difference is negative, which is when
		\texttt{s} is shorter than \texttt{t}.
		\item The error occurs because when an expression contains unsigned operands,
		C will implicitly cast signed operands to unsigned. In this case, the result of \texttt{strlen(s) - strlen(t)} is cast to an unsigned number. Therefore, if the result is negative, it is
		cast to a positive number. If the result is negative, it is then cast to a positive
		number.
		\item We can fix the problem by replacing the expression after the \texttt{return} statement
		with \texttt{return strlen(s) > strlen(t)}.
	\end{enumerate}
\end{sol}
\end{document}