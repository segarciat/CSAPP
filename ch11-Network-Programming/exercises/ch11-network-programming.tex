%%%%%%%%%%%%%%%%%%%%%%%%%%%%%%%%%%%%%%%%%%%%%%%%%%%%%%%%%%%%%%%
% Welcome to the MAT320 Homework template on Overleaf -- just edit your
% LaTeX on the left, and we'll compile it for you on the right.
%%%%%%%%%%%%%%%%%%%%%%%%%%%%%%%%%%%%%%%%%%%%%%%%%%%%%%%%%%%%%%%
% --------------------------------------------------------------
% Based on a homework template by Dana Ernst.
% --------------------------------------------------------------
% This is all preamble stuff that you don't have to worry about.
% Head down to where it says "Start here"
% --------------------------------------------------------------

\documentclass[12pt]{article}

\usepackage{graphicx}
\graphicspath{{./images/}}
\usepackage{textcomp} % cent symbol, such as \textcent
\usepackage[margin=1in]{geometry} 
\usepackage{amsmath,amsthm,amssymb}
\usepackage{cancel}
\usepackage{mathtools} % ceiling function
\DeclarePairedDelimiter{\ceil}{\lceil}{\rceil}
% https://tex.stackexchange.com/questions/146306/how-to-make-horizontal-lists
\usepackage[inline]{enumitem} % allows using letters in enumerate list environment

% source: https://stackoverflow.com/questions/3175105/inserting-code-in-this-latex-document-with-indentation

\usepackage{listings}
\usepackage{color}

\definecolor{dkgreen}{rgb}{0,0.6,0}
\definecolor{gray}{rgb}{0.5,0.5,0.5}
\definecolor{mauve}{rgb}{0.58,0,0.82}

\lstset{frame=tb,
	language=C, % language for code listing
	aboveskip=3mm,
	belowskip=3mm,
	showstringspaces=false,
	columns=flexible,
	basicstyle={\small\ttfamily},
	numbers=none,
	numberstyle=\tiny\color{gray},
	keywordstyle=\color{blue},
	commentstyle=\color{dkgreen},
	stringstyle=\color{mauve},
	breaklines=true,
	breakatwhitespace=true,
	tabsize=4
}

\newcommand{\N}{\mathbb{N}}
\newcommand{\Z}{\mathbb{Z}}

\newenvironment{ex}[2][Exercise]{\begin{trivlist}
		\item[\hskip \labelsep {\bfseries #1}\hskip \labelsep {\bfseries #2.}]}{\end{trivlist}}

\newenvironment{sol}[1][Solution]{\begin{trivlist}
		\item[\hskip \labelsep {\bfseries #1:}]}{\end{trivlist}}


\begin{document}

% --------------------------------------------------------------
%                         Start here
% --------------------------------------------------------------

\noindent Sergio Garcia Tapia \hfill

\noindent{\small Computer Systems: A Programmer's Perspective, by Bryant and O'Hallaron} \hfill

\noindent{\small Chapter 11: Network Programming}

\noindent\today

\subsection*{Practice Problems}

\begin{ex}{11.1}
	Complete the following table:
	\begin{center}
		\begin{tabular}{cc}
			Hex address & Dotted-decimal address \\
			\hline
			\texttt{0x0} & \makebox[1cm]{\hrulefill} \\
			\texttt{0xffffffff} & \makebox[1cm]{\hrulefill} \\
			\texttt{0x7f000001} & \makebox[1cm]{\hrulefill} \\
			\makebox[1cm]{\hrulefill} & \texttt{205.188.160.121}\\
			\makebox[1cm]{\hrulefill} & \texttt{64.12.149.13}\\
			\makebox[1cm]{\hrulefill} & \texttt{205.188.146.23}\\
		\end{tabular}
	\end{center}
\end{ex}

\begin{sol}
	\
	\begin{center}
		\begin{tabular}{cc}
			Hex address & Dotted-decimal address \\
			\hline
			\texttt{0x0} & \texttt{0.0.0.0} \\
			\texttt{0xffffffff} & \texttt{255.255.255.255} \\
			\texttt{0x7f000001} & \texttt{127.0.0.1} \\
			\texttt{0xcdbca079} & \texttt{205.188.160.121}\\
			\texttt{0x400c950d} & \texttt{64.12.149.13}\\
			\texttt{0xcdbc9317} & \texttt{205.188.146.23}\\
		\end{tabular}
	\end{center}
\end{sol}

\begin{ex}{11.2}
	Write a program \texttt{hex2dd.c} that converts its hex argument to a dotted-decimal
	string and prints the results. For example,
	\begin{lstlisting}[language={}]
linux> ./hex2dd 0x8002c2f2
128.2.194.242
	\end{lstlisting}
\end{ex}

\begin{sol}
	\
	My approach was to read the hexadecimal number, the use \texttt{strtol} to with a base
	of 16 as the third argument to parse it as a hexadecimal number in host byte order. I 
	then proceeded to use \texttt{htonl} to convert this from host byte order to network
	byte order and stored it in a variable. I used the address-of operator \texttt{\&} to get
	the pointer to this integer IP address and cast the variable to an \texttt{unsigned char *}
	to interpret it s a pointer to characters, where each character is a byte representation
	of the input number. Finally, I used \texttt{inet\_ntop} to convert it to a dotted-decimal
	string. Notice that the book instead relied on \texttt{sscanf} instead of \texttt{strtol}
	to parse the input. Moreover, the book used the \texttt{struct in\_addr} structure,
	which is the correct data structure for the IP address.
	
	\
	See my approach in \texttt{02-hex2dd/hex2dd.c}:
	\lstinputlisting{02-hex2dd/hex2dd.c}
\end{sol}

\begin{ex}{11.3}
	Write a program \texttt{dd2hex.c} that converts its dotted-decimal argument to a hex number
	and prints the result. For example,
	\begin{lstlisting}[language={}]
linux> ./dd2hex 128.2.194.242
0x8002c2f2
	\end{lstlisting}
\end{ex}

\begin{sol}
	\
	This time I used \texttt{inet\_pton()} to parse the dotted-decimal format IP address,
	and I stored the result in an \texttt{unsigned char []}. I then looped through the
	resulting character array one byte at a time, and I used 4 iterations since an IPv4
	address is made up of 4 bytes. I used the string format \texttt{\%02x} to have a
	hex number with a width of 2 and adding 0s if necessary to pad to the full width.
	This produced the desired output, but the better approach is done in the book, which
	instead of an \texttt{unsigned char []}, they use the \texttt{struct in\_inaddr} structure
	type, which is the correct way to interpret IP address as explained on Section 11.3.1.
	Then looping was not necessary because the entire number can be printed at once by
	using the format specifier \texttt{\%x}.
	
	\
	See my implementation at \texttt{03-dd2hex/dd2hex.c}:
	\lstinputlisting{03-dd2hex/dd2hex.c}
\end{sol}

\end{document}
